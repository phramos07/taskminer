\documentclass[pageno]{jpaper}

\newcommand{\pactedition}{26th}
\newcommand{\pactyear}{17}
\newcommand{\pactcompyear}{20\pactyear}

%replace XXX with the submission number you are given from the PACT submission site.
\newcommand{\pactsubmissionnumber}{XXX}

\usepackage[normalem]{ulem}

\begin{document}

\title{Task Mining in Irregular Programs.}

\date{}
\maketitle

\thispagestyle{empty}

\begin{abstract}
We need to write our abstract here.
\end{abstract}

\section{Introduction}
\label{sec:intro}

The key contribution of this paper is a fully automatic technique to find parallelism in programs, and to annotate them so to make this parallelism explicit.
This technique consists in the combination of static and a dynamic analyses that increase the volume of tasks that can run in parallel.
The static analysis finds independent code, which can be safely executed in parallel.
However, due to its conservative nature, the static analysis misses potential parallelization opportunities.
These tasks, which we cannot show to be independent due, for instance, to poor aliasing information, are solved at runtime by the data-flow engine.
The final product of this combination is a high-performant framework that
lets us run algorithms that are traditionally difficult to parallelize.

\section{Overview}
\label{sec:ovf}

\section{Solution}
\label{sec:sol}

\section{Evaluation}
\label{sec:eval}

\section{Related Work}
\label{sec:rw}

\section{Conclusion}
\label{sec:conc}

\bstctlcite{bstctl:etal, bstctl:nodash, bstctl:simpurl}
\bibliographystyle{IEEEtranS}
\bibliography{references}

\end{document}

