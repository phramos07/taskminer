\documentclass[pageno]{jpaper}

\newcommand{\pactedition}{26th}
\newcommand{\pactyear}{17}
\newcommand{\pactcompyear}{20\pactyear}

%replace XXX with the submission number you are given from the PACT submission site.
\newcommand{\pactsubmissionnumber}{XXX}

\usepackage[normalem]{ulem}

\begin{document}

\title{Task Mining in Irregular Programs.}

\date{}
\maketitle

\thispagestyle{empty}

\begin{abstract}
We need to write our abstract here.
\end{abstract}

\section{Introduction}
\label{sec:intro}

The key contribution of this paper is a technique to run programs in parallel.
This technique consists of a data-flow like hardware,
a runtime environment that detects dependences between data, plus a
static analysis that finds potentially dependent code.
Instructions are scheduled to run in parallel if they do not contain data
dependences, which are detected on the fly by the runtime environment.
However, not every pair of instructions need to be checked -- only those pairs
that a static analysis points as potentially interdependent.
The final product of this combination of hardware, static analysis, runtime checks
and code generator is a high-performant framework that lets us run algorithms that
are traditionally difficult to parallelize.

\section{Overview}
\label{sec:ovf}

\section{Solution}
\label{sec:sol}

\section{Evaluation}
\label{sec:eval}

\section{Related Work}
\label{sec:rw}

\section{Conclusion}
\label{sec:conc}

\bstctlcite{bstctl:etal, bstctl:nodash, bstctl:simpurl}
\bibliographystyle{IEEEtranS}
\bibliography{references}

\end{document}

