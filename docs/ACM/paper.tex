%% For double-blind review submission
\documentclass[sigplan,10pt,review,anonymous]{acmart}
\settopmatter{printfolios=true}
%% For final camera-ready submission
%\documentclass[acmlarge]{acmart}\settopmatter{}

\makeatletter\if@ACM@journal\makeatother

%% Journal information (used by PACMPL format)
%% Supplied to authors by publisher for camera-ready submission
\acmJournal{PACMPL}
\acmVolume{1}
\acmNumber{1}
\acmArticle{1}
\acmYear{2017}
\acmMonth{1}
\acmDOI{10.1145/nnnnnnn.nnnnnnn}
\startPage{1}
\else\makeatother

%% Conference information (used by SIGPLAN proceedings format)
%% Supplied to authors by publisher for camera-ready submission
\acmConference[PL'17]{ACM SIGPLAN Conference on Programming Languages}{January 01--03, 2017}{New York, NY, USA}
\acmYear{2017}
\acmISBN{978-x-xxxx-xxxx-x/YY/MM}
\acmDOI{10.1145/nnnnnnn.nnnnnnn}
\startPage{1}
\fi

\newcommand{\fer}[1]{\textcolor{red}{#1}}

%% For review submission
\setcopyright{none}

%% Bibliography style
\bibliographystyle{ACM-Reference-Format}
\citestyle{acmauthoryear}

%% Packages that we are using in this version of the work:
\usepackage{hyperref}

% To turn comments OFF simply comment out the \Commentstrue line
\newif\ifComments\Commentstrue

\ifComments
\newcommand{\marcio}[1]{\noindent\textcolor{violet}{Marcio: {#1}}}
\newcommand{\guido}[1]{\noindent\textcolor{magenta}{Guido: {#1}}}
\newcommand{\fernando}[1]{\noindent\textcolor{brown}{Fernando: {#1}}}
\newcommand{\cesar}[1]{\noindent\textcolor{magenta}{Cesar: {#1}}}
\newcommand{\pedro}[1]{\noindent\textcolor{brown}{Pedro: {#1}}}
\newcommand{\rmv}[1]{\noindent\textcolor{gray}{Removed: {#1}}}
\newcommand{\new}[1]{\noindent\textcolor{blue}{ {#1}}}
\newcommand{\ed}[1]{\noindent\textcolor{red}{ {#1}}}
\else
\newcommand{\marcio}[1]{}
\newcommand{\guido}[1]{}
\newcommand{\fernando}[1]{}
\newcommand{\cesar}[1]{}
\newcommand{\pedro}[1]{}
\newcommand{\rmv}[1]{}
\newcommand{\new}[1]{#1}
\newcommand{\ed}[1]{}
\fi
\newcommand\dawn{\mbox{\textsf{DawnCC}}}
\newcommand\Taskminer{\mbox{\textsf{TaskMiner}}}

\begin{document}

\title[Automatic Identification and Annotation of Parallel Tasks in Structured
Programs]
{Automatic Identification and Annotation of Parallel Tasks in Programs}

\author{Pedro Henrique Ramos Costa}
\authornote{with author1 note}          %% \authornote is optional;
\orcid{nnnn-nnnn-nnnn-nnnn}
\affiliation{
  \position{Researcher}
  \department{DCC}
  \institution{UFMG}
  \streetaddress{6627 Antonio Carlos Avenue}
  \city{Belo Horizonte}
  \state{Minas Gerais}
  \postcode{31.270-213}
  \country{Brazil}
}
\email{pedroramos@dcc.ufmg.br}

\author{Gleison Souza Diniz Mendonc\c{c}a}
\authornote{with author1 note}          %% \authornote is optional;
\orcid{nnnn-nnnn-nnnn-nnnn}
\affiliation{
  \position{Researcher}
  \department{DCC}
  \institution{UFMG}
  \streetaddress{6627 Antonio Carlos Avenue}
  \city{Belo Horizonte}
  \state{Minas Gerais}
  \postcode{31.270-213}
  \country{Brazil}
}
\email{gleison.mendonca@dcc.ufmg.br}

\author{Divino C\'{e}sar}
\authornote{with author1 note}
\orcid{nnnn-nnnn-nnnn-nnnn}
\affiliation{
  \position{Researcher}
  \department{IC}
  \institution{UNICAMP}
  \streetaddress{Cidade Universit\'{a}ria}
  \city{Campinas}
  \state{S\~{a}o Paulo}
  \postcode{13083-970}
  \country{Brazil}
}
\email{divcesar@gmail.com}

\author{Guido Ara\'{u}jo}
\authornote{with author1 note}
\orcid{nnnn-nnnn-nnnn-nnnn}
\affiliation{
  \position{Professor}
  \department{IC}
  \institution{UNICAMP}
  \streetaddress{Cidade Universit\'{a}ria}
  \city{Campinas}
  \state{S\~{a}o Paulo}
  \postcode{13083-970}
  \country{Brazil}
}
\email{guido@ic.unicamp.br}

\author{Fernando Magno Quint\~{a}o Pereira}
\authornote{with author1 note}
\orcid{nnnn-nnnn-nnnn-nnnn}
\affiliation{
  \position{Professor}
  \department{DCC}
  \institution{UFMG}
  \streetaddress{6627 Antonio Carlos Avenue}
  \city{Belo Horizonte}
  \state{Minas Gerais}
  \postcode{31.270-213}
  \country{Brazil}
}
\email{fernando@dcc.ufmg.br}          %% \email is recommended

\begin{abstract}
We need to write our abstract here.
\end{abstract}

%% 2012 ACM Computing Classification System (CSS) concepts
%% Generate at 'http://dl.acm.org/ccs/ccs.cfm'.
 \begin{CCSXML}
<ccs2012>
<concept>
<concept_id>10011007.10011006.10011041</concept_id>
<concept_desc>Software and its engineering~Compilers</concept_desc>
<concept_significance>500</concept_significance>
</concept>
<concept>
<concept_id>10011007.10011006.10011008.10011024.10011035</concept_id>
<concept_desc>Software and its engineering~Procedures, functions and subroutines</concept_desc>
<concept_significance>300</concept_significance>
</concept>
<concept>
<concept_id>10003752.10003766.10003776</concept_id>
<concept_desc>Theory of computation~Regular languages</concept_desc>
<concept_significance>300</concept_significance>
</concept>
<concept>
<concept_id>10003752.10010124.10010131.10010134</concept_id>
<concept_desc>Theory of computation~Operational semantics</concept_desc>
<concept_significance>300</concept_significance>
</concept>
</ccs2012>
\end{CCSXML}

\ccsdesc[500]{Software and its engineering~Compilers}
\ccsdesc[300]{Software and its engineering~Procedures, functions and subroutines}
\ccsdesc[300]{Theory of computation~Regular languages}
\ccsdesc[300]{Theory of computation~Operational semantics}

\keywords{Parallelism, Tasks, OpenMP}

\maketitle

\section{Introduction}
\label{sec:intro}

Annotation systems have risen to a place of prominence as a simple and
effective way to write parallel programs.
Examples of such systems include OpenMP~\cite{JaegerCP15},
OpenACC~\cite{OpenACC20}, OpenHMPP~\cite{Andion14}, OpenMPC~\cite{Lee10} and OpenSs~\cite{MeenderinckJ11}.
Annotations work as a meta-language: they let developers grant parallel
semantics to syntax originally written to execute sequentially.
The compiler-related literature contains a generous load of success stories
describing the use of annotation systems to parallelize programs.
Combined with modern accelerators, they have led to substantial performance
gains~\cite{Bertolli14,Mendonca17,Poesia17,Reyes12,Wienke12}.
Yet, even if convenient, the use of annotations is not straightforward.

Annotations such as OpenMP or OpenACC still require developers to
worry about typical hassles of the parallel world, such as race conditions,
or deadlocks.
Moreover, because they are used in tandem with imperative programming languages,
such troubles are only made worse by pointer aliasing.
There exist tools that insert automatic annotations in
programs~\cite{gleison2016,Pingali11,wanggenerating}.
All these technologies explore data-parallelism -- the possibility of running
the same computation independently on different data.
Task parallelism remains still uncharted land, in what concerns automatic
annotation.
Such fact is unfortunate, because much of the power of current annotation
systems lays on their ability to create tasks \ed{We need to find references!}.
This ability brings said systems closer to irregular programs such as those
that implement graphs and worklist algorithms \`{a} la
Pingali~\cite{Ben-Nun17,Kulkarni11,Pingali11}.
We address this omission in this work.

This paper describes \Taskminer, the first compiler that mines task parallelism
from programs.
To fulfill this goal \Taskminer{} solves three challenges.
First, it find program regions, e.g., loops or functions, that can be
effectively mapped onto tasks.
Second, it determines symbolic bounds to all the memory blocks accesses within
those regions.
Third, it maps such information back into source code, producing annotations
that programmers can read.
The goal of this paper is to describe the design and the implementation of these
three technologies, and to demonstrate their benefit.

Such benefit is performance and readability.
\Taskminer{} receives as input a C program, and produces, as output, a C
program annotated with OpenMP directives that identify tasks.
We are currently able to annotate non-trivial programs, involving every
sort of composite type available in the C language, e.g., arrays, structs,
unions and pointers to these aggregates.
Some of these programs, such as those taken from the
Kastor~\cite{Virouleau14} or Bots~\cite{Duran09}
benchmark suites, are large and complex; thus, their manual annotation is a
time consuming and error prone task.
Yet, our automatically annotated programs not only approximate the execution
times of the parallel versions of those benchmarks, but are much faster than
their sequential -- unannotated -- versions.

\section{Overview}
\label{sec:ovf}



%EXAMPLE - BFS



\section{Solution}
\label{sec:sol}

\section{Evaluation}
\label{sec:eval}

We aimed to show that our tool exceeded in three main aspects that surround parallel programming. Thus our work evaluation stands on these 3 pillars:

\begin{itemize}

\item Practicality
\item Applicability
\item Performance

\end{itemize}

\textbf{\textit{Practicality}}. Programming in a parallel pattern is deemed difficult by programmers mainly because of the complexity of its data structures and algorithms. We show that our tool is simple and practical, for it does not involve changing the code and data structures in order to fit in a parallel pattern. The code is simply annotated by compiler directives.

\textbf{\textit{Applicability}}. 

\textbf{\textit{Performance}}. When it comes to parallel computing, it is undeniable that the main achievable goal should be performance. There's no reason why programmers would opt for parallelism other than the desire for faster programs. Therefore, we ought to show that the programs annotated with TaskMiner are faster than their serial counterparts. 

\section{Related Work}
\label{sec:rw}

\section{Conclusion}
\label{sec:conc}

\bibliography{references}

\end{document}
